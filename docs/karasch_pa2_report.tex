\documentclass[12pt]{article}

\usepackage{amsmath}
\usepackage{amssymb}
% \usepackage{textcomp}           % For angle brackets in text
\usepackage{centernot}
\usepackage{fancyhdr}
\usepackage{listings}
\lstset{basicstyle=\ttfamily\small,frame=Trbl,numbers=left}
% \usepackage{graphicx}
% \graphicspath{ {images/} }
\setlength{\parskip}{0.5em}


\pagestyle{fancy}
\lhead{Nathan Karasch}
\rhead{COMS 311 - Programming Assignment 2}
\renewcommand{\headrulewidth}{0.4pt}
\renewcommand{\footrulewidth}{0.4pt}


\begin{document}

\section{Data Stuctures for $Q$ and $visited$}

$Q$ was implemented as a \lstinline|LinkedList<String>| object, because
a LinkedList can act as a First-in First-out (FIFO) queue
structure. Vertices can be added to the front and removed from
the end in constant time.

\bigskip
\noindent
$visited$ was implemented as a \lstinline|HashSet<String>| object. Adding
vertices and searching for vertices are the only two operations
needed. Expected runtime for both operations is
$O(h(x) + \text{Average Load})$, where $h(x)$ is the time taken to
hash the string. Assuming the hash function is good (it is) the
average load will be small. Assuming the string lengths are relatively
short and don't grow asymptotically (as is the case for the links),
we can say the time to compute $h(x)$ is negligible. Therefore
the expected runtime for both Insert and Search are approximately
$O(1)$.

\bigskip
\noindent
In short, LinkedList and HashSet were chosen for $Q$ and $visited$
because they both provide constant time operations for the
required operations to implement the algorithm.

\section{Analysis of \textbf{WikiCS.txt}}

\textbf{Number of edges:}

23963

\noindent
\textbf{Number of vertices:}

500

\noindent
\textbf{Vertex with largest out degree:}

/wiki/Computer\_Science (499 outgoing edges)

\noindent
\textbf{Number of strongly connected components:}

9

\noindent
\textbf{Size of the largest component:}

492

\section{Data Structures Used in \lstinline
  [basicstyle=\ttfamily\Large]
  |GraphProcessor|}

\noindent
The main graph was stored in a single
\lstinline|HashMap<String, HashSet<String>>|
object. Each String key is a vertex in the graph, and each key's value
\lstinline|HashSet<String>| is the set of vertices
representing outgoing edges from the key vertex.

\bigskip
\noindent
To compute and store Strongly-Connected Components (SCC), a private
class called \textbf{SCCHelper} was used. It used the following data
structures:

\begin{itemize}
\item \lstinline|HashSet<String> visited|
\item \lstinline|PriorityQueue<VertexTime> finishTimes|
\item \lstinline|HashMap<String, HashSet<String>> reversedGraph|
\item \lstinline|HashSet<String> tempSCC|
\item \lstinline|ArrayList<HashSet<String>> stronglyConnectedComponents|
\end{itemize}

\bigskip
\noindent
The \lstinline|VertexTime| class was just a wrapper class for a String
and an int, representing the vertex and time values to sort vertexes
in the SCC algorithm by decreasing finish times.

\bigskip
\noindent
After the SCC algorithm runs, the SCC's are stored in the\\
\lstinline|stronglyConnectedComponents| variable, and all the other
data structures are set to \lstinline|null| so they can be cleaned up
by the garbage collector and release memory they were using.


\section{Runtimes for \lstinline
  [basicstyle=\ttfamily\Large]
  |GraphProcessor| Methods}

\subsection{outDegree(String v)}

Since the graph only needs to be initialized once, each call to
\lstinline|outDegree(String v)| only takes $O(1)$ runtime because
it simply returns the size of the HashSet at key $v$ in the graph.

\subsection{sameComponent(String u, String v)}

Iterates over each \lstinline|HashSet<String>| in the pre-computed
strongly-connected components ArrayList and checks if
$u$ and $v$ are in the HashSet. If the ArrayList contains
$k$ HashSets, the runtime of
\lstinline|sameComponent(String u, String v)| is $O(k)$.

\bigskip
\noindent
Since $k \leq n$, (where $n$ is the number of vertices in the graph),
The worst-case runtime in terms of graph vertices is $O(n)$.

\subsection{componentVertices(String v)}

Iterates over each \lstinline|HashSet<String>| in the pre-computed
strongly-connected components ArrayList and checks if
$v$ is in the HashSet. The HashSet containing $v$ is converted to
an ArrayList, which takes $O(j)$ time for a HashSet of size $j$.
Worst-case runtime happens if $v$ is in the last HashSet checked.
The runtime for the method is then $O(k + j)$, where $k$ is the
number of HashSet objects in the SCC ArrayList.

\bigskip
\noindent
Since $k + j \leq n + 1$, (where $n$ is the number of vertices
in the graph), The worst-case runtime in terms of graph vertices
is $O(n)$.

\subsection{largestComponent()}

Iterates over each \lstinline|HashSet<String>| in the pre-computed
strongly-connected components ArrayList, looks at the size of
each, and returns the max size. If the ArrayList contains
$k$ HashSets, the runtime for the method is $O(k)$.

\bigskip
\noindent
Since $k \leq n$, (where $n$ is the number of vertices in the graph),
The worst-case runtime in terms of graph vertices is $O(n)$.

\subsection{numComponents()}

Returns the size of the pre-computed SCC ArrayList, which can be
done in constant time.

\subsection{bfsPath(String u, String v)}

The algorithm looks roughly as follows:

\begin{enumerate}
\item Initialize an ArrayList $A$
\item Create a BFS-Tree starting at $u$
\item Set $current$ to $v$ (in the BFS-Tree)
\item While parentOf($current$) is not null, add $current$ to $A$
  and set $current$ to parentOf($current$).
\item Reverse the list $A$ before returning it, so it goes
  from $u$ to $v$
\item Return $A$
\end{enumerate}

\noindent
Creating a BFS-Tree takes $O(m + n)$ time, where $m$ is the number
of edges in the graph and $n$ is the number of vertices. Going up the
BFS-Tree from $v$ to $u$ takes $O(n)$ time, and reversing the list
takes $O(n)$ time. Therefore, the \lstinline|bfsPath()| method runs
in $O(m + n)$ time.

\end{document}