\documentclass[12pt]{article}

\usepackage{amsmath}
\usepackage{amssymb}
\usepackage{centernot}
\usepackage{fancyhdr}
\usepackage{listings}
\lstset{basicstyle=\ttfamily\small,frame=Trbl,numbers=left}
% \usepackage{graphicx}
% \graphicspath{ {images/} }
\setlength{\parskip}{0.5em}


\pagestyle{fancy}
\lhead{Nathan Karasch}
\rhead{COMS 311 - Programming Assignment 2}
\renewcommand{\headrulewidth}{0.4pt}
\renewcommand{\footrulewidth}{0.4pt}


\begin{document}

\section{Data Stuctures for $Q$ and $visited$}

$Q$ was implemented as

Give choice and rationale

\section{Analysis of \textbf{WikiCS.txt}}

\textbf{Number of edges:}

23963

\noindent
\textbf{Number of vertices:}

500

\noindent
\textbf{Vertex with largest out degree:}

/wiki/Computer\_Science (499 outgoing edges)

\noindent
\textbf{Number of strongly connected components:}

9

\noindent
\textbf{Size of the largest component:}

492

\section{Data Structures Used in \lstinline
  [basicstyle=\ttfamily\Large]
  |GraphProcessor|}

\section{Runtimes for \lstinline
  [basicstyle=\ttfamily\Large]
  |GraphProcessor| Methods}

\subsection{outDegree(String v)}

\subsection{sameComponent(String u, String v)}

\subsection{componentVertices(String v)}

\subsection{largestComponent()}

\subsection{numComponents()}

\subsection{bfsPath(String u, String v)}






\noindent\makebox[\linewidth]{\rule{\paperwidth}{0.4pt}}








\section{Algorithm for
  \lstinline[basicstyle=\ttfamily\Large]
  |buildDataStructure()|}

\begin{lstlisting}
  public void buildDataStructure() {
    int n = (int) Math.floor(1.5 * points.size());
    hashTable = new HashTable(n);
    for (float f : points) {
      // Use the neighbor-preserving hash as the
      // tuple key
      hashTable.add(new Tuple(npHash(f), f));
    }
  }
\end{lstlisting}

The neighbor-preserving hash, as specified in the guidelines, is the
floor function. \\

\begin{lstlisting}
  private static int npHash(float p) {
    return (int) Math.floor(p);
  }
\end{lstlisting}

In general terms, the algorithm looks like so: \\

\begin{enumerate}
\item Create a hash table of size $floor(1.5 * n)$, where $n$ is the
  number of points you will put into your data structure. The 1.5 is
  chosen to avoid the hash table reaching a load factor of 0.7
  as points get added. So it will never have to waste time doubling
  size as the hash table is built.
\item For each float in the set of points, create a Tuple with the
  float as the value $v$ and the neighbor-preserving hash as the
  key $k$. The neighbor-preserving hash chosen was $k = floor(v)$.
  \item Add the Tuple($k$, $v$) to the hash table.
  \end{enumerate}

  

\pagebreak
\section{Algorithm for
  \lstinline[basicstyle=\ttfamily\Large]
  |npHashNearestPoints(float p)|}

\begin{lstlisting}
  public ArrayList<Float> npHashNearestPoints(float p) {
    ArrayList<Float> result = new ArrayList<>();
    int npHashP = npHash(p);

    ArrayList<Tuple> candidateTuples
                            = hashTable.search(npHashP);

    // Points with the exact same npHash are close
    for (Tuple t : candidateTuples) {
      result.add(t.getValue());
    }

    // Points with npHashes that differ by +/- 1 MIGHT
    // be close (requires checking)
    candidateTuples = hashTable.search(npHashP + 1);
    candidateTuples.addAll(
                     hashTable.search(npHashP - 1));
    for (Tuple t : candidateTuples) {
      if (areClose(p, t.getValue())) {
        result.add(t.getValue());
      }
    }

    return result;
  }
\end{lstlisting}

In general terms, the algorithm looks like so:

\begin{enumerate}
\item Search the hash table for all elements with key $k$ matching the
  neighbor-preserving hash $h$ of the input $p$. Add these to
  a results array $A$, since anything with the same neighbor-preserving
  hash will be close to $p$.
\item Search the hash table for all elements where $k = h+1$ or $k=h-1$.
  These elements \textit{might} be close to $p$, but we need to check
  to verify.
\item For each candidate tuple with key $h+1$ or $h-1$, check to see
  if the actual value is close to $p$, and add ``close'' points to $A$.
  \item Return $A$.
\end{enumerate}




\pagebreak
\section{Actual Runtimes}

Commands were executed on a \lstinline|NearestPoints| instance loaded
with the 100,000 points given in the ``points.txt'' file. The runtime
measurement of \lstinline|new NearestPoints("points.txt")|
includes the cost of \lstinline|buildDataStructure()|, but the other
methods do not. All runtimes were measured using the following method: \\

\begin{lstlisting}
  long start = System.currentTimeMillis();
  executeCommand();
  long end = System.currentTimeMillis();
  long runtime = end - start;
\end{lstlisting}
\bigskip

\textbf{new NearestPoints("points.txt")}: 242 ms \\

\textbf{allNearestPointsNaive()}: 44050 ms \\

\textbf{allNearestPointsHash()}: 170 ms


\end{document}